\documentclass[11pt, letterpaper]{article}
\usepackage[]{hyperref}
\usepackage{graphicx}
\bibliographystyle{apalike}
\title{Lab rotation report: Escape behaviour in mice}
\author{Max Bardelang}
\date{December 2025}

\begin{document}
% TITLE PAGE -----------------------------------------------------------

\begin{center}
	\thispagestyle{empty}
	
	\LARGE{Graduate School of Computational Neuroscience}\\
	\LARGE{University of Tübingen}\\
	
	\vspace{3cm}
	
	\LARGE{\textbf{Escape behaviour in mice}}\\[-0.5ex]
	
	\vspace{6cm}
	
	\large{Laboratory report \par}
	\large{presented by \par}
	\large{Max Bardelang}
	
	\vspace{1.6cm}
	
	The study was supervised by \par
	Prof. Dr. Burgalossi, Florian Hofmann\par
	Werner Reichardt Centre for Integrative Neuroscience\\
	Neural Circuits and Behavior group\\
		
	\vspace{1.6cm}
	
	Duration of the lab rotation: 9 weeks \par
	Deadline for submission: 15.12.2025
	
\end{center}

\pagebreak


% ABSTRACT ---
\begin{abstract}
	The anterior dorsal nucleus plays a important role in relaying sensory information. Its exact function however is unclear.
\end{abstract}
\thispagestyle{empty}
\newpage

\tableofcontents
\thispagestyle{empty}
\newpage

% INTRODUCTION
\section{Introduction}
% "chain of reasoning that led to the question or hypothesis your study addresses"
% Whats in the introduction
This aim of this project is to pilot a experimental environment and analysis pipeline for a planned study on head direction cells in the anterior dorsal nucleus. The neuroscientific background of this study will be introduced concisely to provide a background for understanding the experimental design.


\subsection{Head direction cells of the anterior dorsal nucleus}
% From neuroscientific background to experimental design
% HD cells in anterior thalamic nucleus

% Goal of piloted study
%Goal: Whether and how the anterior dorsal nucleus is involved in spatial episodic memory formation.
%To asses, whether and how the head direction cells of the anterior dorsal nucleus is involved in spatial episodic memory formation

Neurons of the anterios dorsal nucleus are known to contain a subset of cells that are head direction (HD) tuned and form part of a cortical circuit that plays a central role in episodic memory formation. These HD cells are known to provide directional input to downstream hippocampal neurons, functionally linked to memory impairments. In a recent study, \cite{Blanco-Hernndez2024} revealed additional correlations of these HD cells with sensory and behavioural input, that non-HD-tuned neurons in the anterior dorsal nucleus did not exhibit. Concretely,..,.suggests,... . Based on these correlations, a follow-up study now will investigate the \textit{functional} relationship of these HD cells with episodic memory formation by selectively inactivating them by optogenetic tools and resulting behavioural effects in mice. %maybe explain why ethological relevance is important
The goal of this project to prepare the experimental setup and analysis pipeline. A cruicial parameter in this study is the ethological relevance under which this episodic memory formation takes place. Escape behaviour in rodents is such a behaviour and will be defined in the following paragraph.



\subsection{Escape behaviour in mice}
% What is escape
Escape behaviour in mice \cite{BrancoEscapeBehavuour2025} is a rapid, stereotyped and ethologically relevant defensive action that is triggered when a sensory stimulus signals imminent danger. Behaviourally, escape typically consists of a brief latency followed by an abrupt initiation of high-speed locomotion, often preceded by a short orienting or freezing phase. Once escape is initiated, mice generate a directed movement that increases distance from the perceived threat and, when available, orientation toward a place of safety. In laboratory paradigms, animals reorient their head and body axis toward a previously learned shelter location, even when the shelter is not visible at the time of threat, indicating that spatial information is rapidly accessed at escape onset. After onset, escape is sustained until the animal reaches shelter, highlighting that escape behaviour is not a reflexive turn-and-run, but a temporally structured sequence comprising fast initiation, goal-directed reorientation, and continued locomotion until safety is achieved.
\\


\subsection{Project description and goal}
%Reorientation accuracy is measured as the difference between the head direction vector and the homing vector of the mouse after reorientation during escape.

The success of the project will be evaluated along two mayor requirements:

First, a environment has to be created, in which episodic memory formation can be studied by initiating and recording escape behaviour in mice. A experimental paradigm closely related is used in \cite{Campagner2023}. Escape behaviour has to be reproduced under this paradigm, with the additional introduction of a modification as will be described down below. Second, a analysis pipeline has to be implemented, that takes these recordings as input and extracts head direction and homing vector along the time axis of individual trials. To allow scaling the analysis to extended periods of time and to multiple animals, pose tracking of these bodyparts has to be automated, to which a Convolutional Neural Network (CNN) for computer vision is employed. These results will also allow evaluating qualitatively, whether mice in the prepared experimental setup exhibit the behaviour described in the preceding paragraph.



% MATERIALS AND METHODS
\section{Materials and Methods}


\subsection{Experimental setup}
% Arena
All trials were performed in a elevated circular arena made of solid plastic of ~90cm in diameter. In order to constrain the mouse to accessing the memorized shelter location and prevent accessing visual information to estimate the shelter location, it was placed as an underground shelter, barred from sight when the mouse is out in the arena. To achieve this, the shelter was mounted at the border of the arena, so the upper edge aligned with the arena surface, in a modification of the original paradigm used in \cite{Campagner2023}.


%Experimental setup
Data was recorded using Spike2 (Cambridge Electronic Design Limited), which controlled camera and speakers. Sound stimuli were send by the program after a button press from the experimenter and reached sound pressure levels of 65-80 dB at the centre of the arena. TTL triggers for controlling frames were sent to the camera at a frame rate of 25 per second. Spike2 recordings were exported and converted to .mat format. The camera lens was changed beginning with recordings of mice 5, which resulted in a drop in image resolution due to unresolved hardware issues.


\subsection{Mice}

\subsection{Pose tracking}
Pose tracking of left ear, right ear and nose was performed using the software package DeepLabCut 3.0.0rc9(DLC) \cite[]{Nath2019} for Python 3.10 to train a CNN for pose estimation. Code with detailed descriptions can be found in \autoref{A}.\\

A training data set of 200 uncropped images, selected by DLC's automatic k-means frames extraction method from recordings of the first three mice, was selected and manually labeled. Importantly, this training set does not yet include mice under the second camera lens introduced in recordings of subsequent mice. DLC offes pretrained weight initialisations with corresponding models fitted to different scenarios of pose tracking. The weight initialisation pretrained to a overhead view of mice (SuperAnimal-TopViewMouse)\cite{Ye_2024} was choosen, in combination with a underlying HRNet-w32 architecture and the default image augmentation method albumentations.\\


Labeling data was performed on a desktop-PC, due to the requirement of a graphical user interface. All subsequent steps for pose tracking were run on the Cluster of the Werner Reichardt Centre for Integrative Neuroscience, where model fine tuning was performed using a NVIDIA A100 40GB.\\


Pose tracking on a recording yields both the estimated coordinates by the model as well the confidence of the model in this estimation as a scalar in the range of [0, 1].
% Maybe look up threshold below which DLC puts points to generic posistion at corner.

% Synchronizing trigger data and video recordings



\subsection{Homing vector analysis}
Homing vector analysis was run using Python 3.10 and pandas 2.3.3. Trials were segmented to contain 1000ms prestimulus and 5000 ms poststimulus intervals, and the corresponding number of frames (6s * 25 frames/s = 150 frames) were extracted per trial. Shelter position was manually determined for each mouse, as its exact pixel position within the frame changed slightly between recording sessions of different mice.

At each timepoint, head center position was calculated as the midpoint between the ears, head direction as the vector from head centre to nose, and homing vector as the vector from head centre to shelter position. The absoloute angular difference (AAD) was defined as the absoloute value of the angle between head direction vector and homing vector, in degrees´.

To select valid trials, angular velocity was determined as the first derivative of the AAD and plotted for each trial. Trials with a notable deflection in angular velocity after stimulus presentation were selected as valid trials, as such a deflection reflects a rapid movement of the head of the mouse, i.e. a reorientation - possibly towards the shelter.

To asses reorientation accuracy, the AAD of selected valid trials was calculated and plotted - together with the trajectories of the individual valid trials, along the time axis of a trial.

% RESULTS
\section{Results}

\subsection{Pose estimation}
\subsubsection{Evaluating estimation quality}
The evaluation of the confidence assigned by the model to its own estimations show medium quality of pose estimation data. These results in \ref{fig:confidencehistograms} from the model trained on data of the first three mice show a clear decline in pose estimation quality after mouse 5, where switch the camera lens caused a change in image resolution.


Figures in Appendix \ref{A} additionally highlight large gaps in the pose estimation data during crucial timespans after stimulus onset for some trials.

\begin{figure}
	\centering
	\includegraphics[width=.9\linewidth]{confidence_histograms}
	\caption[short short]{Pose estmation confidence. Histograms of confidence score (likelihood) given to the pose estimation in each frame within a trial by DeepLabCut model. Performance drop pronounced in fifth and sixth mice.}
	\label{fig:confidencehistograms}
\end{figure}


\newpage
% DISCUSSION
\section{Discussion}

The drop in pose estimation confidence underscores, that this pipeline for pose estimation is not robust to changes in the recording setup. To achieve a reliably high quality of estimations, it is suggested to train another pose estimation model with data recorded under the final recording setup, possibly also utilizing DeepLabCuts additional functionality for active learning.

% ACKNOWLEDGEMENTS

% LITERATURE
\newpage
\bibliography{./references}

\appendix
\section{DeepLabCut commands}\label{A}
Initializing the dataset has to be run on the cluster as well, as models are downloaded during this step.

\section{Pose estimation confidence}
\begin{figure}
	\centering
	\includegraphics[width=0.7\linewidth]{confidence_mouse1}
	\caption{Pose estimation confidence of single trials in mouse 1. See \ref{fig:confidencemouse6} for description. Upper 10 are trials with highest average confidence, bottom 10 are trials with lowest average confidence.}
	\label{fig:confidencemouse1}
\end{figure}

\begin{figure*}
	\centering
	\includegraphics[width=0.65\linewidth]{confidence_mouse6}
	\caption{Pose estimation confidence of single trials (trials with and without escape behaviour, before valid trials selection) in mouse 6. Estimation confidence is lowest confidence among pose estimations for all three bodyparts per frame. One tile contains one trials.  With frame no° (i.e. time) on x-axis. Green vertical bar indicated stimulus onset. Red horizontal bar as visual guidance towards marking threshold for very low confidence of 0.33 out of possible 1.0 .}
	\label{fig:confidencemouse6}
\end{figure*}

\end{document}
